%!TEX root = main.tex
\section{Data}

To predict time of module execution we create a machine learning model for each module. With each execution of the module, within some application, we will get another data point to train our model. We will use the following features (explanatory variables)\footnote{As a docker container that will be used to execute a module do not use SWAP memory, RAM is not colerated with execution time. In this program we will not use modules with GPU support. Summarazing, the GPUs and RAM resources limits are not taken into account for modeling.} as an input data for the model:
\begin{enumerate}
	\item mCPU limit - called \textit{mili cores} - the fraction of a physical CPU used to carry out the module execution,
	\item total size of an input data,
	\item max element size of an input data (if the input data is a set of files type it is the biggest part of data to be processed),
	\item avg size of an input element,
	\item number of input data elements.
\end{enumerate}
This last three features makes sense only if the input data is a set of files. Otherwise, if the data is just a single file (e.g. data frame), the features set should be reduced to just the first two elements from the list above.

Obviously, our dependent value (that we are going to estimate) is an execution time of a module. The example data frame that will be used to train and validate our models is presented in the table~\ref{tab:example_df}.
\begin{table*}[!t]
	\centering
	\caption{\label{tab:example_df}The example data frame for models training and validations.}
	\begin{minipage}{0.9\linewidth}
	{\footnotesize
		\begin{tabular}{|c c c c c c >{\columncolor[gray]{0.9}}c|} 
			\hline
			Module ID & mCPU & total size [KB] & max size [KB] & average size [KB] & number of elements & time [ms] \\ [0.5ex] 
			\hline\hline
			1 & 1.1 & 12414 & 1341 & 871 & 21 & 7813 \\ 
			\hline
			1 & 0.5 & 12414 & 1341 & 871 & 21 & 12406  \\
			\hline
			1 & 3.6 & 54001 & 2190 & 891 & 82 & 9017 \\
			\hline
			... & ... & ... & ... & ... & ... & ... \\ [1ex] 
			\hline
		\end{tabular}
	}
	\end{minipage}
\end{table*}	
