%%%%% Fichero de ejemplo LaTeX que ilustra el uso de la Hoja de Estilo %%%%%%
%%%%% Jornadas.cls para Jornadas Sarteco

\documentclass[twocolumn,twoside]{Jornadas}
\usepackage[utf8]{inputenc}
\usepackage[english]{babel}
\usepackage{listings}
\usepackage{algorithm}
\usepackage{algorithmicx}
\usepackage{algcompatible}
\usepackage{adjustbox}
\usepackage{graphicx}
\usepackage{color}
\usepackage{caption}
\captionsetup{font=footnotesize}
\usepackage[caption=false,font=footnotesize]{subfig}
%\setlength{\marginparwidth}{2cm}
%\usepackage{todonotes}
\usepackage{placeins}
\usepackage{hyperref}
\usepackage{url}
\usepackage{msc}
\usepackage{xcolor,colortbl}
\usepackage{float}

\newcommand{\subtitle}[1]{%
  \posttitle{%
    \par\end{center}
    \begin{center}\large#1\end{center}
    \vskip0.5em}%
}

\makeatletter
% \getcountref extracts the raw reference and
% results zero for undefined references.
% Caution: undefined references will not generate
% LaTeX warnings!
\newcommand*{\getcountref}[1]{%
	\expandafter\@getcountref\csname r@#1\endcsname
}
\newcommand*{\@getcountref}[1]{%
	\ifx#1\relax
	0% undefined reference
	\else
	\expandafter\@car#1\@empty\@nil
	\fi
}
\makeatother

\DeclareRobustCommand*{\citeinside}{\cite}

% -*- Ajustes LaTeX en relación a las figuras -*-
\setcounter{topnumber}{10}     % Max. numero de figs. on top
\setcounter{bottomnumber}{10}  % Max. numero de figs. abajo
\setcounter{totalnumber}{10}   % Max. numero de figs. por pagina
\renewcommand{\topfraction}{1} % Max. fraccion de pagina ocupada por figs.
\renewcommand{\bottomfraction}{1}
\renewcommand{\textfraction}{0}  % Min. fraccion de pagina ocupada por texto
\renewcommand{\floatpagefraction}{1} % Max. espacio de pagina solo con figs.

\definecolor{gray97}{gray}{.97}
\definecolor{gray75}{gray}{.75}
\definecolor{gray45}{gray}{.45}

\lstset{
     inputencoding=utf8,
     extendedchars=true,
     backgroundcolor=\color{gray97},
     %
     stringstyle=\ttfamily,
     showstringspaces = false,
     basicstyle=\scriptsize\ttfamily,
     commentstyle=\color{gray45},
     keywordstyle=\bfseries,
     linewidth=.98\columnwidth,
     xleftmargin=3mm,
     breaklines=true,
     numbers=left,
     numbersep=6pt,
     numberstyle=\tiny,
     numberfirstline = false,
     firstnumber=auto,
     breaklines=true,
     %The worst-case complexity of an algorithm is the greatest number of operations needed to solve the problem over an input data of a given size
     escapeinside={(*@}{@*)},
     literate={á}{{\'a}}1 {é}{{\'e}}1 {í}{{\'i}}1 {ó}{{\'o}}1 {ú}{{\'u}}1 {ñ}{{\~n}}1
   }


\colorlet{punct}{red!60!black}
\definecolor{background}{HTML}{EEEEEE}
\definecolor{delim}{RGB}{20,105,176}
\colorlet{numb}{magenta!60!black}

\lstdefinelanguage{json}{
	linewidth=7.5cm,
	basicstyle=\small\ttfamily,
	numbers=left,
	numberstyle=\scriptsize,
	stepnumber=1,
	numbersep=8pt,
	showstringspaces=false,
	breaklines=false,
	frame=lines,
	backgroundcolor=\color{background},
	literate=
	*{0}{{{\color{numb}0}}}{1}
	{1}{{{\color{numb}1}}}{1}
	{2}{{{\color{numb}2}}}{1}
	{3}{{{\color{numb}3}}}{1}
	{4}{{{\color{numb}4}}}{1}
	{5}{{{\color{numb}5}}}{1}
	{6}{{{\color{numb}6}}}{1}
	{7}{{{\color{numb}7}}}{1}
	{8}{{{\color{numb}8}}}{1}
	{9}{{{\color{numb}9}}}{1}
	{:}{{{\color{punct}{:}}}}{1}
	{,}{{{\color{punct}{,}}}}{1}
	{\{}{{{\color{delim}{\{}}}}{1}
	{\}}{{{\color{delim}{\}}}}}{1}
	{[}{{{\color{delim}{[}}}}{1}
	{]}{{{\color{delim}{]}}}}{1},
}

\def\BibTeX{{\rm B\kern-.05em{\sc i\kern-.025em b}\kern-.08em
    T\kern-.1667em\lower.7ex\hbox{E}\kern-.125emX}}

\newtheorem{theorem}{Teorema}

\hyphenation{pa-ra-le-lis-mo pro-cee-dings}

%Directorios en los que se buscan las figuras
\graphicspath{{.}{./figures/}}
%%%%%%%%%%%%%%%%%%%%%%%%%%%%%%%%%%%%%%%%%%%%

\begin{document}

\font\myfont=cmr12 at 22pt
\title{\myfont
	Run-time estimation for data processing tasks \\
}

\author{%
     Jan Bielecki%
}

\maketitle
% Oculta las cabeceras y los números de página.
% Ambos elementos se añadirán durante la edición de las actas completas.
\markboth{}{}
\pagestyle{empty} 
\thispagestyle{empty} % Oculta el número de la primera página

\begin{abstract}
The aim of the research is to estimate the run-time of a data processing task (specific execution of a program or an algorithm) before the run. In the article we do not strictly focus on providing an estimation of maximum execution time (WCET\footnote{Worst-case execution time - the maximum amount of time that the execution can take.}). We try to estimate the average-case execution time (ACET) which is absolutely enough for this use case requirements. The chosen approach is to create machine learning models using historical or testing data of the executions. Each execution of a program has the set of explanatory variables as properties of an input data and execution environment. Obviously, response variable is the run-time of a program.
\end{abstract}

\begin{keywords}
multicore computing, data processing, analysis of algorithms, run-time estimation, nonlinear regression, small-p regression, small-n regression, machine learning, KNN, SVR, RBF
\end{keywords}

%Añade los ficheros que necesites:

%!TEX root = main.tex
\section{Introduction}
\subsection{Context}

The research described in the article is focus on providing the tool for estimating the run-time of a single CAL module (data processing program or algorithm). Computation Application Language (CAL) is designed to write large scale application in due to perform big data processing. Each CAL program consists of modules that are separetely executed in sequence (with parallelization possibility) on virtual machines and sending the results futher to a next module of the application. CAL language is a part of the system developed under the Baltic LSC\cite{baltic_lsc_website} project.

During the data processing within the application run, each module is executed with some input data. It could be different kind of data e.g. data frame or set of images. The execution of a module is invoke on a virtual machine with limited resources (RAM, mCPUs, GPUs). The properties of an input data and the execution environment resources will be used to estimate the overall application execution time and price on a specific cluster.

Baltic Large Scale Computing (BalticLSC\cite{baltic_lsc}) project is using CAL language to perform data processing tasks. Each task is an execution of a CAL application. Each module can be used in many CAL applications. We can say that a single module is a one block (commonly as a docker image) and an application is the schema of executing a sequence of blocks. Figure~\ref{fig:face_recogniser_app} shows the example application that consists of a few modules to perform a face recognition on the input video data. Some set of modules within a CAL application can be executed in parallel. Some modules can be also run as a multiple instances of itself to make data processing faster.

The worst-case complexity of an algorithm is the greatest number of operations needed to solve the problem over an input data of a given size. The analysis of algorithmic complexity emerged as a scientific subject during the 1960’s and has been quickly established as one of the most active fields of study\cite{complexity}. The most common way to describe an algorithm complexity is the \textit{big O} notation (collectively called Bachmann–Landau notation or asymptotic notation) which is the universal formula that describe how the run-time of an algorithm grow as the input data size grows. 

In our work, we study the formula (more precise and complex one in comparison to the \textit{big O} notation) that takes mixed sets of input data and run-time environment properties as arguments. Moreover, the models we introduce in further parts, will provide the estimation of the data processing program run-time and not only the complexity formula as it the \textit{big O} notation does. 

\begin{figure*}[!t]
	\centering
	\begin{minipage}{0.9\linewidth}
		\includegraphics[width=1.0\textwidth]{face_recogniser_app}
	\end{minipage}
	\caption{\textit{Face Recogniser} scheme. Application written in the CAL language.}
	\label{fig:face_recogniser_app}
\end{figure*}

\subsection{Related work}

The articles listed above contain content that to some extent coincides with the research carried out in this article:

\begin{itemize}
	\item \textit{Estimation of the execution time in real-time systems}\cite{wcet} - an overview on the different approaches to estimate the program execution time. The authors focus on the \textit{WCET} concept. The aim for this project is not to estimate execution time as a deadline that could not be exceed. We try to estimate the average-case execution time (\textit{ACET}).

\item \textit{Execution Time Analysis for Embedded Real-Time Systems}\cite{time-analysis} - as is said in the section 5. of the article, a static timing  analysis tool works by statically analysing the properties of the program (embedded to a specific structure like vector or graph) that affect its timing behavior. Our approach, on the other hand, is to statically analyze the properties of input data and the run-time environment resources.

\item \textit{Using online worst-case execution time analysis and alternative tasks in real time systems}\cite{images-processing-time} - similar approach to carry out the regression using input data feature (section 3.6). The authors use an amount of image pixels as the only one feature to estimate the execution time of few different image processing algorithm.

\item \textit{A Prediction Model for Measurement-Based Timing Analysis}\cite{ga} - the authors made en experiment by generating random lists of data with the same properties and then train the artificial neural network to predict the \textit{WCET}. They used \textit{Gem5} to simulate the run-time environment. We have used the docker as an execution environment which allows to receive more real data with a natural noise introduce to the response variable (execution time). The authors develop their work in the next article\cite{surogate} using the concept of surogate models as a solution to the problem of generating training data. Such generating can increase overheads of the execution time estimation for processing algorithms with heavy input data.

\item \textit{Nonlinear approach for estimating WCET during programming phase}\cite{program-features} - the authors estimate a \textit{WCET} by extracting the program features from object code. It starts when the source code of a program is successfully compiled into object code. Then, the extracted features was used for subsequent samples optimization and \textit{WCET} estimate. The authors use SVR algorithm with RBF kelner as we did likewise.

\item \textit{A solution to minimum sample size for regressions}\cite{small-n} - the authors explore the problem of a small amount of training data which also affects the result of our article. Testing data generation and simple algorithm for the run-time estimation are used to reduce the impact of a small n problem on the final results of our work.

\end{itemize}

\subsection{Contribution}
Using simple machine learning algorithms (\textit{SVR} and \textit{KNN}, more about the algorithms and how they fit to the problem in the further sections) to estimate a program run-time. 

Determine the basic properties of an input data and run-time environment for data processing task. 

Creating a separate model for each module (program) instead of embedding the program structure to a vector of features or control flow graphs (CFGs).

%!TEX root = main.tex
\section{Data}

We created a machine learning model for each module to estimate the run-time of a CAL module execution. We will get another data point to train our model with each module's run (within some application of the Baltic LSC system). The following features (explanatory variables)  make up the input data set for a model:

\begin{enumerate}
	\item CPUs limit - called \textit{mili-cores} - the fraction of a physical CPU used to carry out the module execution.
	\item The total size of an input data in bytes.
	\item The largest element size of an input data (if the input data is a set of files type, it is the largest size of a file within the set).
	\item The average size of an input element.
	\item The number of input data elements.
\end{enumerate}

The last three features make clear sense only if input data is a set of files. Otherwise, if data is just a single file, the features can be a sort of data features representation carrying more detailed information about the data than only a total size. For the data frame type, the number of input elements can be equal to the number of columns. The max element size will be a quotient of the total size and the number of columns, and the average size will be equal to that quotient as well.  In other cases (other kinds of input data), one can prepare specific, additional features and store their values for each execution (to enable their use for training in the future).  In this work, we simplified the task to the features mentioned above. Our dependent value (that we are going to estimate) is an execution time of a module. 

We created two CAL applications based on four modules with different types of input data. The first application (consisting of three modules) takes the movie as input data, marks people's faces on each frame, and then returns the movie with marked people's faces as output data. The second one consists of just a single module, and it searches for the best hyperparameters of the XGBoost algorithm within the parameters grid. Table~\ref{tab:modules} describes the modules that we used in the research. For each module, we created a set of 10 different input data sets. Next, we ran the modules with all the mentioned data sets using various CPU resources (from 0.5 CPUs up to 4.0 CPUs with 0.5 step). Finally, we received the data frame with the number of 4 (number of modules) * 8 (different CPUs resources) * 10 (number of input data sets) = 320 rows that we used to train and validate our models. Part of the models' input data is presented in Table~\ref{tab:example_df}. 

\begin{figure*}[!t]
	\centering
	\begin{minipage}{0.75\linewidth}
		\includegraphics[width=1.0\textwidth]{corr}
	\end{minipage}
	\caption{Data columns Spearman`s correlation per module.}
	\label{fig:corr}
\end{figure*}

\begin{figure*}[!t]
	\centering
	\begin{minipage}{0.75\linewidth}
		\includegraphics[width=1.0\textwidth]{pca_importance}
	\end{minipage}
	\caption{PCA importance per module.}
	\label{fig:pca}
\end{figure*}

\begin{table*}[!t]
	\centering
	\caption{\label{tab:example_df}Part of the data frame for models training and validation.}
	\begin{minipage}{0.9\linewidth}
		{\footnotesize
			\begin{tabular}{|c c c c c c >{\columncolor[gray]{0.9}}c|} 
				\hline
				Module ID & mCPUs & total size [B] & number of elements & max size [B] & average size [B] & time [s] \\ [0.5ex] 
				\hline\hline
				1 & 4.0 & 1703379 & 544 & 3131 & 3131 & 3.909 \\ 
				\hline
				1 & 4.0 & 809881 & 548 & 1477 & 1477 & 1.981  \\
				\hline
				1 & 4.0 & 1711796 & 1392 & 1229 & 1229 & 11.371 \\
				\hline
				... & ... & ... & ... & ... & ... & ... \\ [1ex] 
				\hline
			\end{tabular}
		}
	\end{minipage}
\end{table*}	

To make the models' data more understandable, we prepared two figures that show interactions between the features. Figure~\ref{fig:corr} shows how each column of the dataset is correlated with the other ones (green indicates positive correlation, red indicates negative correlation). As a correlation coefficient between two columns, we used Spearman's rank correlation coefficient. As we can see, the data-based input features are mostly positively correlated with each other to some point. They obviously are positively correlated with the run-time as well. The CPUs feature is negatively correlated with the run-time for each module (highest correlation for \textit{face\_recogniser}). As the data-based input features are strongly correlated, we performed the PCA analysis to determine how the specif column affects the input data variation. On the \textit{x}-axis of Figure 3, starting from the left with the component with the highest impact on the data variance, we can see the PCA results. The colors on the Figure show the share of features in each of the components.  CPUs' only environment-based feature is not correlated with other ones and impacts the data variation with a value equal to 1/(number of columns) = 0.2 for each module. The rest of the variance falls on data-based features. The PCA analysis shows that we could compress our input data to only three components and still keep almost all variance (the same situation for each module). We kept the original input data for more clearness, but one should consider dimensionality reduction if he has decided to provide many features.

\begin{table}[hbt!]
	\centering
	\caption{\label{tab:modules}Modules that we used in the research.}
	\begin{tabular}{|c c c|} 
		\hline
		ID (APP ID) & Name & Input data \\ [0.5ex] 
		\hline\hline
		1(1) & video\_splitter & video file \\ 
		\hline
		2(1) & face\_recogniser & image files \\
		\hline
		3(1) & images\_merger & image files \\
		\hline
		4(2) & xgb\_grid\_search & CSV file \\
		\hline
	\end{tabular}
\end{table}

%!TEX root = main.tex
\section{Algorithms}
\subsection{Support Vector Regression}

As a one of machine learning algorithms to estimate time of a CAL module execution we will use \textit{Epsilon-Support Vector Regression}~\cite{svrc} (SVR) - it is based on Support Vector Machine (SVM, orginally named \textit{support-vector networks}\cite{svm}).

We will use SVR model with a RBF kernel\cite{rbf_kernel} which is the most known flexible kernel and it could project the features vectors into infinite dimensions. It uses Taylor expansion which is equivalent to use an infinite sum over the polynomial kernels. It allows to model any function that is a sum of unknown degree polynomials.

Using kernel, the resulting algorithm is formally similar, except that every dot product is replaced by a nonlinear kernel function. This allows the algorithm to fit the maximum-margin hyperplane in a transformed feature space. The RBF kernel have the following form:

\[ K_{RBF}(\vec{x}, \vec{x}') = \exp{(-\gamma||\vec{x} - \vec{x}'||)}\ \textnormal{, where:}\]
\begin{itemize}
	\item $ ||\vec{x} - \vec{x}'|| $ is the squared Euclidean distance between the two vectors of features,\newline
	\item $ \gamma $ - hyperparameter described in more details below.
\end{itemize}
In the SVR algorithm we are looking for a hyperplane \textit{y} in the following form:
\[ y = \vec{w} \vec{x} + b \textnormal{, where:}\]
\begin{itemize}
	\item $ \vec{x} $ - vector of features,
	\item $ \vec{w} $ - normal vector to the hyperplane \textit{y}, using a kernel the $ \vec{w} $ is also in the transformed space.
\end{itemize}
Training the original SVR means solving:
\[ \frac{1}{2}||w||^2  + C \sum_{i}^{N}(\xi_i+\xi_i^*) \textnormal{,}\]
with the following constraints:
\[ y_i - \vec{w}x_i - b \le \epsilon + \xi_i  \]
\[ -y_i + \vec{w}x_i + b \le \epsilon + \xi_i^*  \]
\[ \xi_i \xi_i^* \ge 0 \]
Figure~\ref{fig:svrc} shows the wanted hyperplane with marked \textit{$\xi$} and \textit{$\epsilon$} parameters.
As we already choosed the RBF kernel for the SVR algorithm, our modelling is simplified to just find the best values of the following hyperparameters\cite{svr}:
\begin{enumerate}
	\item \textit{C} -the weight of an error cost. The regularization\footnote{Regularization is a way to give a penalty to certain models (usually overly complex ones)} hyperparameer, have to be strictly positive. The example from the figure use l1 penalty (the library that we use to modelling use the squared epsilon-insensitive loss with l2 penalty). The strength of the regularization is inversely proportional to C. The larger value of C the more variance is introduced into the model. 
	\item \textit{epsilon} - It specifies the epsilon-tube within which no penalty is associated in the training loss function with points predicted within a distance epsilon from the actual value. As it is shown of the figure~\ref{fig:svr_epsilon_gamma} the green data points do not provide any penalty to the loss function because they are within the allowed epsilon range around the approximation\footnote{\label{figure_example}The figure is based on some example data and it only shows the hyperparameter influence on model.}.
	\item \textit{gamma} - The \textit{gamma} hyperparameter can be seen as the inverse of the radius of influence of samples selected by the model as support vectors. Increasing the value of \textit{gamma} hyperparameter causes the variance increase what is shown in the figure~\ref{fig:gamma}\footnotemark[\getcountref{figure_example}].
	
\end{enumerate}
\begin{figure}[!htb]
	\caption{Visualization of the SVR's epsilon and gamma parameters~\citeinside{epsilon}.}
	\centering
	\label{fig:svr_epsilon_gamma}
	\includegraphics[width=0.5\textwidth]{svr_epsilon_gamma}
\end{figure}
\begin{figure}[!htb]
	\caption{SVR's gamma hyperparameter~\citeinside{gamma} influence on the model variance.}
	\centering
	\label{fig:gamma}
	\includegraphics[width=0.45\textwidth]{svr_gamma}
\end{figure}
You can find detailed information about C and gamma params in sklearn library documentation\cite{rbf_params}.

\subsection{K-nearest Neighbors Regression}

Another, examined algorithm used to estimating time of a module execution is a simple \textit{KNN}\footnote{\textit{KNN} - k-nearest neighbors} regression. Using the algorithm, a target is predicted by local interpolation of the targets associated of the nearest neighbors in the training set~\cite{knnreg}.In another words, the target assigned to a query point is computed based on the mean of the targets of its nearest neighbors.

To use the algorithm we have to define values of the following parameters:
\begin{enumerate}
	\item \textit{k} - number of the nearest neighbors
	\item \textit{weights} - weight function used in prediction. The basic nearest neighbors regression uses uniform weights: that is, each point in the local neighborhood contributes uniformly to the classification of a query point. Under some circumstances, it can be advantageous to weight points such that nearby points contribute more to the regression than faraway points. The weights can be calculated from the distances using any function for example the linear one.
	\item \textit{algorithm} - the procedure to calculate k-nearest neighbors for the query point. It does not have a direct impact to the final regression result, but the parameters of the algorithm surely have. For example, using the BallTree algorithm we have to choose the \textit{metric} parameter that will be used to calculate the distance between data points. It could be, for example, the Minkowski~\cite{minkowski} metric with the l2 (standard Euclidean~\cite{euclidean}) distance metric or any function that will calculate the distance between two points.
\end{enumerate}

\begin{figure*}[!t]
	\centering
	\begin{minipage}{0.45\linewidth}
		\includegraphics[width=1.0\textwidth]{svr_out_of_range}
	\end{minipage}
	\begin{minipage}{0.45\linewidth}
		\includegraphics[width=1.0\textwidth]{knn_out_of_range}
	\end{minipage}
	\caption{Regression on the out of training range data.}
	\label{fig:out_of_range}
\end{figure*}
%!TEX root = main.tex
\section{Training and validation}
\begin{figure*}[!t]
	\centering
	\begin{minipage}{0.85\linewidth}
		\includegraphics[width=1.0\textwidth]{learning_curve_multi}
	\end{minipage}
	\caption{Learning curve per module.}
	\label{fig:curve}
\end{figure*}

The training pipeline for each module contains the following steps:
\begin{enumerate}
	\item Having the 80 data points, we divide them into training and test data sets with 1:4 proportion.
	\item Standardization of a data set is a common requirement for many machine learning estimators: they might behave badly if the individual features do not more or less look like standard normally distributed data (e.g. Gaussian with 0 mean and unit variance)\cite{scaler}. We scale each column (feature) of the data set using the following formula:
	\[ \vec{x}_{f}^{'} = \frac{\vec{x}_{f}-\mu_f}{\sigma_f} \textnormal{, where:}\]
	\begin{itemize}
		\item $ \vec{x}_{f} $ - data set column \textit{f},
		\item $ \mu_f $ - mean of the column \textit{f},
		\item $ \sigma_f $ - standard deviation of the column \textit{f}.
	\end{itemize}
	\item Each algorithm have a few parameters that should be chosen wisely in order to the better results. It is hard to predict the best value of a continuous parameter. What we did is an exhaustive search over specified parameter values for an estimator from the given possible values. For each combination of the parameter values we validate the model using 5-fold cross validation (as it was mentioned in the first step). It is called a \textit{grid search}\cite{grid_search}.
	\item Finally, we retrained our model using the full data set and the parameters that were found in the previous step.
	\subsection{Support Vector Regression}
	The parameters grid for the \textit{SVR} algorithm is listed below:
	\begin{lstlisting}[language=json,firstnumber=1]
'gamma': [0.0001, 0.0002, 0.0004, 0.0008, 
0.0016, 0.0032, 0.0064, 0.0128], 
'epsilon': [1e-06, 2e-06, 4e-06, 8e-06, 
1.6e-05, 3.2e-05, 6.4e-05, 0.000128, 
0.000256, 0.000512, 0.001024], 
'C': [1000.0, 2000.0, 4000.0, 8000.0,
16000.0, 32000.0, 64000.0, 128000.0, 
256000.0, 512000.0, 1024000.0, 2048000.0]
	\end{lstlisting}
	
	\subsection{K-nearest Neighbors}
	The parameters grid for the \textit{KNN} algorithm is listed below:
	\begin{lstlisting}[language=json,firstnumber=1]
'n_neighbors': [1, 2, 3, 4, 5, 6, 7, 8, 9, 10, 11], 
'weights': ['uniform', 'distance'], 
'p': [1, 2]
	\end{lstlisting}
	
	\begin{figure*}[!t]
		\centering
		\begin{minipage}{0.45\linewidth}
			\includegraphics[width=1.0\textwidth]{svr_xgb_grid_search_surf}
		\end{minipage}
		\begin{minipage}{0.45\linewidth}
			\includegraphics[width=1.0\textwidth]{knn_xgb_grid_search_surf}
		\end{minipage}
		\caption{Regression surfaces for \textit{xgb\_grid\_search} module.}
		\label{fig:curve}
	\end{figure*}

\begin{figure*}[!t]
	\centering
	\begin{minipage}{0.45\linewidth}
		\includegraphics[width=1.0\textwidth]{svr_images_merger_surf}
	\end{minipage}
	\begin{minipage}{0.45\linewidth}
		\includegraphics[width=1.0\textwidth]{knn_images_merger_surf}
	\end{minipage}
	\caption{Regression surfaces for \textit{images\_merger} module.}
	\label{fig:curve}
\end{figure*}

\begin{figure*}[!t]
	\centering
	\begin{minipage}{0.45\linewidth}
		\includegraphics[width=1.0\textwidth]{svr_video_splitter_surf}
	\end{minipage}
	\begin{minipage}{0.45\linewidth}
		\includegraphics[width=1.0\textwidth]{knn_video_splitter_surf}
	\end{minipage}
	\caption{Regression surfaces for \textit{video\_splitter} module.}
	\label{fig:curve}
\end{figure*}

\begin{figure*}[!t]
	\centering
	\begin{minipage}{0.45\linewidth}
		\includegraphics[width=1.0\textwidth]{svr_face_recogniser_surf}
	\end{minipage}
	\begin{minipage}{0.45\linewidth}
		\includegraphics[width=1.0\textwidth]{knn_face_recogniser_surf}
	\end{minipage}
	\caption{Regression surfaces for \textit{face\_recogniser} module.}
	\label{fig:curve}
\end{figure*}
		
	\subsection{Final results}
	
	\begin{table*}[!t]
		\centering
		\caption{\label{tab:example_df}The final results.}
		\begin{minipage}{1\linewidth}
			{\footnotesize
				\begin{tabular}{|c| c >{\columncolor[gray]{0.9}}c| c c |} 
					\hline
					Algorithm: & SVR & & KNN & \\
					\hline
					& best params & relative error [\%] & best params & relative error [\%] \\ [0.5ex] 
					\hline\hline
					\textit{video\_splitter} & \textit{C}: 256000.0, \textit{$\gamma$}: 1e-04, \textit{$\epsilon$}: 2e-06 & 3131 & \textit{n}: 2, \textit{weights}: 1e-04, \textit{p}: 2e-06 & 3.909 \\ 
					\hline
					\textit{face\_recogniser} & \textit{C}: 256000.0, \textit{$\gamma$}: 1e-04, \textit{$\epsilon$}: 2e-06 & 3131 & \textit{n}: 2, \textit{weights}: 1e-04, \textit{p}: 2e-06 & 3.909 \\
					\hline
					\textit{xgb\_grid\_search} & \textit{C}: 256000.0, \textit{$\gamma$}: 1e-04, \textit{$\epsilon$}: 2e-06 & 3131 & \textit{n}: 2, \textit{weights}: \textit{uniform}, \textit{p}: 1 & 49.3 \\
					\hline
					\textit{images\_merger} & \textit{C}: 256000.0, \textit{$\gamma$}: 1e-04, \textit{$\epsilon$}: 2e-06 & 3131 & \textit{n}: 2, \textit{weights}: 1e-04, \textit{p}: 2e-06 & 3.909 \\
					\hline
				\end{tabular}
			}
		\end{minipage}
	\end{table*}	
\end{enumerate} 


\subsection{Training}
...
\subsection{Validation}
...
%!TEX root = main.tex
\section{Conclusions}

As the \textit{PCA} analysis shows (see the Figure~\ref{fig:pca} and its description), the input data for our models have approximately three independent components that contain all variations of the data. This means that by using these components, we will reduce the model to a three-dimensional feature space only. We did not decide to use reduced space, but one can use the reduction to the more complex case because each model can be extended with additional features based on the specific module. For example, there could be some additional parameters required to run a module, and the parameters have an impact on the execution time. These parameters should be included in the set of input features for the module`s model.

Analyzing only the final results, one can see that the used algorithms give relatively similar results with a bit of advantage for the \textit{SVR} (average error of 38.0\%, comparing to 40.4\% for the \textit{KNN}). The more critical characteristic which makes the \textit{SVR} model more promising in use is the possibility to estimate the run-time in the more general way for the inputs out from the training range (see Figure~\ref{fig:out_of_range} and its description). 

In the subject of estimating the execution time of the whole application, the work that was done is not enough. It has to be noticed that we must also predict the input data properties for the next modules in the queue because, at the start, we only know the input data properties for the starting module of the application. This is true for applications consisting of more than one module. It will be implemented in further work to make the possibility to estimate the time of whole application execution.



%\nocite{*}
\bibliographystyle{Jornadas}
\bibliography{biblio}

\end{document}
